% Chapter Template

\chapter{Conclusiones} % Main chapter title

\label{Chapter5} % Change X to a consecutive number; for referencing this chapter elsewhere, use \ref{ChapterX}


%----------------------------------------------------------------------------------------

%----------------------------------------------------------------------------------------
%	SECTION 1
%----------------------------------------------------------------------------------------

\section{Conclusiones generales }

El prototipo desarrollado corresponde a un sistema capaz de realizar el reconocimiento de usuarios previamente registrado, mediante la verificación del patrón característico de su huella digital o mediante la utilización de una clave numérica. El sistema permite crear una base de datos con los usuarios y visualizar un historial de reconocimientos.

Durante el desarrollo de este trabajo se aplicaron los conocimientos adquiridos a lo largo de la carrera logrando de esta forma alcanzar los siguientes objetivos:

\begin{itemize}
\item Desarrollar sistemas embebidos sobre plataformas con microcontroladores de 32 bits y el uso de periféricos.

\item Utilizar buenas practicas de programación sobre el lenguaje C.

\item Gestionar proyectos mediante la elaboración de planes, calendarios, herramientas y metodologías de ingeniería.

\item Utilizar Linux como sistema operativo para sistemas embebidos.

\item Aplicar criterios sobre protocolos de comunicación para elaborar bibliotecas modulares para la abstracción de hardware.

\item Iniciar en el desarrollo de sistemas embebidos con el uso de herramientas de software libre.

\item Diseñar e implementar interfaces gráficas para usuarios.

\item Iniciar en el desarrollo web para registro de acontecimientos sobre sistemas distribuidos.
\end{itemize}

%----------------------------------------------------------------------------------------
%	SECTION 2
%----------------------------------------------------------------------------------------
\section{Próximos pasos}

Cociente de las potencialidades del sistema y para dar continuidad al esfuerzo realizado, se listan a continuación las principales lineas para un trabajo futuro apuntando al desarrollo de un producto comercialmente atractivo.

\begin{itemize}
\item Mejorar la interfaz web mediante el uso de formularios para mejorar la gestión de usuarios.

\item Implementar toda la funcionalidad del interfaz gráfico en la sección de configuraciones y mejorar todo el conjunto.

\item Ampliar la funcionalidad del sistema para cubrir un mayor número de sensores.

\item Implementar todas las funcionalidades para la librería del sensor de huella.

\item Aplicar el sistema en un proyecto de seguridad real.

\item Implementar herramientas para testing de software.

\item Implementar herramientas y métodos para seguridad de datos de usuario.

\item Reemplazar el modelo lo cliente servidor local por el modelo global a fin que el cada parte del sistema pueda ser implementado bajo cualquier red y en cualquier lugar geográfico.

\end{itemize}
