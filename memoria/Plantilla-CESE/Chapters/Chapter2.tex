\chapter{Introducción Específica} % Main chapter title

\label{Chapter2}

%----------------------------------------------------------------------------------------
%	SECTION 1
%----------------------------------------------------------------------------------------
En esta sección se presenta el contexto general en el cual se realizó el diseño, los elementos que conforman el sistema, la descripción de los requerimientos base y la presentación de las diferentes fases mediante las que se abordó la realización del proyecto.

Además se establece una perspectiva global de la ejecución mediante la planificación seguida durante el desarrollo.
  
\section{Visión general de los elementos constitutivos}
%------------------------------------------------------------------------------------
El sistema implementado se centra en una placa de desarrollo embebida, un módulo-sensor lector de huellas y una pantalla para interacción con el usuario.

La placa de desarrollo embebida puede ser definida como un sistema electrónico basado en un microprocesador diseñado para realizar funciones dedicadas.

En el caso del módulo-sensor de adquisición de huella digital es un dispositivo capaz de capturar la imagen de la huella, procesarla y almacenarla para su posterior utilización.

Para una fácil manipulación del sistema por parte del usuario, se incorpora una interfaz gráfica desplegada sobre una pantalla táctil.

En el caso del software, se selecciona como base una distribución del sistema operativo Linux para sistemas embebidos.

\subsection{Descripción funcional}
La única vía para acceder al sistema por parte del usuario es a través de la interfaz gráfica la cual cuenta con un menú  de tres partes principales:

\begin{itemize}
\item Menú para acceso biométrico (acceso por reconocimiento de huella).	
\item Menú para acceso mediante validación de clave numérica personal.
\item Menú para configuraciones.
\end{itemize}

De esta forma el usuario puede acceder únicamente si su huella  o clave esta registrada con anterioridad. Cabe mencionar que la asignación de claves y registro de usuarios es privilegio del usuario principal.

En caso de ser reconocida la huella o clave del usuario, se activa una salida digital de la placa de desarrollo (que en el futuro comandará un actuador) y se despliega en pantalla un mensaje de notificación de bienvenida o caso contrario un mensaje de usuario no reconocido.

El usuario principal tiene además la posibilidad de acceder a una página web donde se guarda el registro de los últimos accesos.
La figura \ref{fig:funcional} muestra la secuencia funcional del sistema.

\begin{figure}[h]
	\centering
	\includegraphics[scale=.9]{./Figures/funcional.pdf}
	\caption{Diagrama de secuencia del sistema.}
	\label{fig:funcional}
\end{figure}

Para lograr este objetivo se diseñó un firmware con el diagrama de bloques de la figura\ref{fig:bloques} .

\begin{figure}[H]
	\centering
	\includegraphics[scale=.6]{./Figures/bloques.pdf}
	\caption{Diagrama de bloques de la arquitectura de firmware diseñada.}
	\label{fig:bloques}
\end{figure}

 
Al desarrollar un sistema modular de este tipo se logra convertir la aplicación de un sensor de reconocimiento de huella en un posible sistema distribuido de seguridad, resultando en un sistema más completo. Resulta relativamente sencillo anexar mas clientes TCP que pueden a su vez manejar nuevas interfaces gráficas con nuevos sensores y con distintas funcionalidades.

%------------------------------------------------------------------------------------
\section{Requerimientos}
%------------------------------------------------------------------------------------
A continuación se presenta el detalle de los requerimientos.

1.- Requerimientos del sistema:

\begin{itemize}
\item 1.1. Interfaz de usuario simple.	
\item 1.2.El sistema posee dos modos de operación principales, biométrico y password.
\item 1.3. En modo password se realiza la identificación del usuario mediante la validación de una clave numérica de cuatro dígitos.
\item 1.4. En modo biométrico se realiza la dentificación del usuario mediante la validación del patrón característico de la huella digital correspondiente al dedo indice de la mano derecha.
\item 1.5. El sistema permite a un usuario identificado como principal, modificar la base de datos del sistema permitiendo crear nuevos usuarios o borrarlos.
\item 1.6. El usuario identificado como principal tiene la opción de acceder a un registro de actividad que le informe que usuario y a que hora tuvo acceso.
\item 1.7. El sistema implementa el modo de acceso permitido como la activación temporal de una salida digital de la placa de desarrollo y un mensaje en pantalla.
\end{itemize}

%------------------------------------------------------------------------------------
\section{Planificación}
%------------------------------------------------------------------------------------
A continuación se muestra el desglose de tareas del proyecto.

1. Planificación (35h).
\begin{itemize}
\item Realizar plan del proyecto.(15h)
\item Realizar el análisis de factibilidad.(10h)
\item Realizar la gestión de calidad.(10h)
\end{itemize} 
2. Investigación Preliminar.(105 hs)
\begin{itemize}
\item Buscar información sobre sistemas de control de acceso por parámetros biométricos.(10 hs)
\item Buscar información sobre módulos de adquisición de huellas dactilares.(10 hs)
\item Buscar información sobre módulos display con tecnología touch.(10 hs)
\item Buscar información acerca de entornos para programación de GUIs.(15 hs)
\item Buscar información sobre bibliotecas para el uso de pantallas touch.(20 hs)
\item Buscar info. sobre bibliotecas para módulos de adquisición de huellas dactilares.(20 hs)
\item Buscar información sobre aplicaciones web para monitoreo remoto.(20 hs)
\end{itemize}

3. Selección de módulos comerciales y plataformas para el proyecto.(130 hs)
\begin{itemize}
\item Selección y pruebas preliminares del módulo y display touch.(30 hs)
\item Selección y pruebas preliminares del módulo de adquisición de huellas dactilares.(30 hs) 
\item Selección y pruebas preliminares del entorno de programación de GUIs.(45 hs)
\item Selección y pruebas para aplicación web para monitoreo remoto. (25hs)
\end{itemize}

4. Desarrollo de la interfaz de usuario (GUI).(40 hs)
5. Desarrollo de firmware.(200h)
\begin{itemize}
\item Implementar las funciones para la obtención de datos desde el módulo biométrico.(30 hs)
\item Implementar funciones para el desarrollo del módulo de monitoreo.
\item Corrección de errores.
\end{itemize}
6. Integración del sistema. (100h)
7. Procesos finales. (100h)

El diagrama Activity on Node final del proyecto se observa en la figura \ref{fig:none}, en ésta se puede apreciar el orden para la ejecución de las tareas, las cuales en total suman 710 horas.

Puede notarse que el camino crítico requiere una inversión de 535 horas.

\begin{figure}[h]
	\centering
	\includegraphics[scale=.8]{./Figures/none.pdf}
	\caption{Diagrama Activity on Node.}
	\label{fig:none}
\end{figure}

%\subsection{Uso de mayúscula inicial para los título de secciones}

%Si en el texto se hace alusión a diferentes partes del trabajo referirse a ellas como capítulo, sección o subsección según corresponda. Por ejemplo: ``En el capítulo \ref{Chapter1} se explica tal cosa'', o ``En la sección \ref{sec:ejemplo} se presenta lo que sea'', o ``En la subsección \ref{subsec:ejemplo} se discute otra cosa''.

%Entre párrafos sucesivos dejar un espacio, como el que se observa entre este párrafo y el anterior. Pero las oraciones de un mismo párrafo van en forma consecutiva, como se observa acá. Luego, cuando se quiere poner una lista tabulada se hace así:

%\begin{itemize}
%	\item Este es el primer elemento de la lista.
%	\item Este es el segundo elemento de la lista.
%\end{itemize}

%Notar el uso de las mayúsculas y el punto al final de cada elemento.

%Si se desea poner una lista numerada el formato es este:

%\begin{enumerate}
%	\item Este es el primer elemento de la lista.
%	\item Este es el segundo elemento de la lista.
%\end{enumerate}

%Notar el uso de las mayúsculas y el punto al final de cada elemento.

%\subsection{Este es el título de una subsección}
%\label{subsec:ejemplo}

%Se recomienda no utilizar \textbf{texto en negritas} en ningún párrafo, ni tampoco texto \underline{subrayado}. En cambio sí se sugiere utilizar \textit{texto en cursiva} donde se considere apropiado.

%Se sugiere que la escritura sea impersonal. Por ejemplo, no utilizar ``el diseño del firmware lo hice de acuerdo con tal principio'', sino ``el firmware fue diseñado utilizando tal principio''. En lo posible hablar en tiempo pasado, ya que la memoria describe un trabajo que ya fue realizado.

%Se recomienda no utilizar una sección de glosario sino colocar la descripción de las abreviaturas como parte del mismo cuerpo del texto. Por ejemplo, RTOS (\textit{Real Time Operating System}, Sistema Operativo de Tiempo Real) o en caso de considerarlo apropiado mediante notas a pie de página.

%Si se desea indicar alguna página web utilizar el siguiente formato de referencias bibliográficas, dónde las referencias se detallan en la sección de bibliografía de la memoria,utilizado el formato establecido por IEEE en \citep{IEEE:citation}. Por ejemplo, ``el presente trabajo se basa en la plataforma EDU-CIAA-NXP, la cual se describe en detalle en \citep{CIAA}''.

%\subsection{Figuras} 

%Al insertar figuras en la memoria se deben considerar determinadas pautas. Para empezar, usar siempre tipografía claramente legible. Luego, tener claro que es incorrecto escribir por ejemplo esto: ``El diseño elegido es un cuadrado, como se ve en la siguiente figura:''

%\begin{figure}[h]
%\centering
%\includegraphics[scale=.35]{./Figures/cuadradoAzul.png}
%\end{figure}

%La forma correcta de utilizar una figura es la siguiente: ``Se eligió utilizar un cuadrado azul para el logo, el cual se ilustra en la figura \ref{fig:cuadradoAzul}''.

%\begin{figure}[h]
%	\centering
%	\includegraphics[scale=.35]{./Figures/cuadradoAzul.png}
%	\caption{Ilustración del cuadrado azul que se eligió para el diseño del logo.}
%	\label{fig:cuadradoAzul}
%\end{figure}

%El texto de las figuras debe estar siempre en español, excepto que se decida reproducir una figura original tomada de alguna referencia. En ese caso la referencia de la cual se tomó la figura debe ser indicada en el epígrafe de la figura e incluida como una nota al pie, como se ilustra en la figura \ref{fig:palabraIngles}.

%\begin{figure}[h!]
%	\centering
%	\includegraphics[scale=.25]{./Figures/word.jpeg}
%	\caption{Imagen tomada de la página oficial del procesador\protect\footnotemark.}
%	\label{fig:palabraIngles}
%\end{figure}

%\footnotetext{\url{https://goo.gl/images/i7C70w}}


%La figura y el epígrafe deben conformar una unidad cuyo significado principal pueda ser comprendido por el lector sin necesidad de leer el cuerpo central de la memoria. Para eso es necesario que el epígrafe sea todo lo detallado que corresponda y si en la figura se utilizan abreviaturas entonces aclarar su significado en el epígrafe o en la misma figura.

%\begin{figure}[h]
%	\centering
%	\includegraphics[scale=.4]{./Figures/questionMark.png}
%	\caption{El lector no sabe por qué de pronto aparece esta figura.}
%	\label{fig:questionMark}
%\end{figure}

%Nunca colocar una figura en el documento antes de hacer la primera referencia a ella, como se ilustra con la figura \ref{fig:questionMark}, porque sino el lector no comprenderá por qué de pronto aparece la figura en el documento, lo que distraerá su atención.

%\subsection{Tablas}

%Para las tablas utilizar el mismo formato que para las figuras, sólo que el epígrafe se debe colocar arriba de la tabla, como se ilustra en la tabla \ref{tab:peces}. Observar que sólo algunas filas van con líneas visibles y notar el uso de las negritas para los encabezados.  La referencia se logra utilizando el comando \verb|\ref{<label>}| donde label debe estar definida dentro del entorno de la tabla.

%\begin{verbatim}
%\begin{table}[h]
%	\centering
%	\caption[caption corto]{caption largo más descriptivo}
%	\begin{tabular}{l c c}    
%		\toprule
%		\textbf{Especie}       & \textbf{Tamaño}  & \textbf{Valor aprox.}\\
%		\midrule
%		Amphiprion Ocellaris	  & 10 cm 			& \$ 6.000 \\		
%		Hepatus Blue Tang      & 15 cm			 & \$ 7.000 \\
%		Zebrasoma Xanthurus    & 12 cm			 & \$ 6.800 \\
%		\bottomrule
%		\hline
%	\end{tabular}
%	\label{tab:peces}
%\end{table}
%\end{verbatim}

%\begin{table}[h]
%	\centering
%	\caption[caption corto]{caption largo más descriptivo}
%	\begin{tabular}{l c c}    
%		\toprule
%		\textbf{Especie} 	 & \textbf{Tamaño}  & \textbf{Valor aprox.}  \\
%		\midrule
%		Amphiprion Ocellaris	 & 10 cm 			& \$ 6.000 \\		
%		Hepatus Blue Tang	 & 15 cm				& \$ 7.000 \\
%		Zebrasoma Xanthurus	 & 12 cm				& \$ 6.800 \\
%		\bottomrule
%		\hline
%	\end{tabular}
%	\label{tab:peces}
%\end{table}

%En cada capítulo se debe reiniciar el número de conteo de las figuras y las tablas, por ejemplo, Fig. 2.1 o Tabla 2.1, pero no se debe reiniciar el conteo en cada sección. Por suerte la plantilla se encarga de esto por nosotros.

%\subsection{Ecuaciones}
%\label{sec:Ecuaciones}

%Al insertar ecuaciones en la memoria estas se deben numerar de la siguiente forma:

%\begin{equation}
%	\label{eq:metric}
%	ds^2 = c^2 dt^2 \left( \frac{d\sigma^2}{1-k\sigma^2} + \sigma^2\left[ d\theta^2 + \sin^2\theta d\phi^2 \right] \right)
%\end{equation}
                                                        
%Es importante tener presente que en el caso de las ecuaciones estas pueden ser referidas por su número, como por ejemplo ``tal como describe la ecuación \ref{eq:metric}'', pero también es correcto utilizar los dos puntos, como por ejemplo ``la expresión matemática que describe este comportamiento es la siguiente:''

%\begin{equation}
%	\label{eq:schrodinger}
%	\frac{\hbar^2}{2m}\nabla^2\Psi + V(\mathbf{r})\Psi = -i\hbar \frac{\partial\Psi}{\partial t}
%\end{equation}

%Para las ecuaciones se debe utilizar un tamaño de letra equivalente al utilizado para el texto del trabajo, en tipografía cursiva y preferentemente del tipo Times New Roman o similar. El espaciado antes y después de cada ecuación es de aproximadamente el doble que entre párrafos consecutivos del cuerpo principal del texto. Por suerte la plantilla se encarga de esto por nosotros.

%Para generar la ecuación \ref{eq:metric} se utilizó el siguiente código:

%\begin{verbatim}
%\begin{equation}
%	\label{eq:metric}
%	ds^2 = c^2 dt^2 \left( \frac{d\sigma^2}{1-k\sigma^2} + 
%	\sigma^2\left[ d\theta^2 + 
%	\sin^2\theta d\phi^2 \right] \right)
%\end{equation}
%\end{verbatim}

%Y para la ecuación \ref{eq:schrodinger}:

%\begin{verbatim}
%\begin{equation}
%	\label{eq:schrodinger}
%	\frac{\hbar^2}{2m}\nabla^2\Psi + V(\mathbf{r})\Psi = 
%	-i\hbar \frac{\partial\Psi}{\partial t}
%\end{equation}

%\end{verbatim}